\documentclass{article}
\usepackage[utf8]{vietnam}

\title{svm}
\author{cuongvomanh}
\date{March 2019}

\usepackage{natbib}

\begin{document}
    \section{Proven}
    % \begin{enumerate}
    \begin{description}
        \item[1] Phụ đề $(*)$
          
        Ta có:
        Vector từ điểm $O(0,0)$ đến đường thẳng $\vec{w}\vec{x} + w_0 = 0 (d)$ 
        hay vector từ điểm $O(0,0)$ đến hình chiếu của điểm $O$ lên 
        đường thẳng $d$ là:
        $$\vec{(O,d)} = \frac{-w_0}{||\vec{w}||}$$

        Từ đây ta có:
        Vector từ điểm $I$ đến đường thẳng $\vec{w}\vec{x} + w_0 = 0 (d)$ là:
        
        $$\vec{(I,d)} = -\vec{(O,d')} + \vec{(O,d)}$$
        Với $\vec{w}(\vec{x} - \vec{I}) = 0 (d')$ 
        là đường thẳng đi qua $I$ và song song với đường thẳng $d$
        $$\Rightarrow \vec{(I,d)} = -\frac{\vec{w}\vec{I}}{||\vec{w}||} +  \frac{-w_0}{||\vec{w}||}$$
        $$\Rightarrow \vec{(I,d)} = -\frac{\vec{w}\vec{I}+w_0}{||\vec{w}||}$$

        Từ đó khoảng cách từ điểm $I$ đến đường thẳng $\vec{w}\vec{x} + w_0 = 0 (d)$ là:
        $$d(d,O) = ||\frac{\vec{w}\vec{I}+w_0}{||\vec{w}||}|| $$
        

        \item [2] Phụ đề $(**)$
        
        Luôn chọn được một cách biểu diễn cho một đường thẳng bất kỳ
        có dạng $\vec{w}\vec{x} + w_0 = 0 (d)$ sao cho hai đường thẳng song song 
        và cách đều đường thẳng $d$ một khoảng cách bất kỳ có dạng:
        $$\vec{w}\vec{x} + w_0 - 1 = 0 (d_1)$$
        $$\vec{w}\vec{x} + w_0 + 1 = 0 (d_2)$$

        \textbf{Chứng minh} \newline
        Giả sử hai đường thẳng $d_1$ và $d_2$ có dạng:
        $$\vec{w}\vec{x} + w_0 - m = 0 (d_1)$$
        $$\vec{w}\vec{x} + w_0 + m = 0 (d_2)$$
        Giả sử khoảng cách giữa $d_1$ và $d_2$ là $\Delta _{12}=p$.
        Theo Phụ đề $(*)$ ta có khoảng cách giữa $d_1$ và $d_2$ là:
        $\Delta _{12} = \frac{2m}{||\vec{w}||}$ 
        $\Rightarrow m = \frac{p||\vec{w}||}{2}$
        Vậy để khoảng cách giữa $d_1$ và $d_2$ là $p$ thì $d_1$ và 
        $d_2$ có dạng:
        $$\vec{w}\vec{x} + w_0 - \frac{p||\vec{w}||}{2} = 0 (d_1)$$
        $$\vec{w}\vec{x} + w_0 + \frac{p||\vec{w}||}{2} = 0 (d_2)$$

        hay 
        $$\frac{2\vec{w}}{p||\vec{w}||}\vec{x} + \frac{2w_0}{p||\vec{w}||} - 1 = 0 (d_1)$$
        $$\frac{2\vec{w}}{p||\vec{w}||}\vec{x} + \frac{2w_0}{p||\vec{w}||} + 1 = 0 (d_2)$$
        và phương trình đường thẳng $d$ có thể được viết lại thành:
        $$\frac{2\vec{w}}{p||\vec{w}||}\vec{x} + \frac{2w_0}{p||\vec{w}||} = 0$$

        Đặt $\vec{w'} = \frac{2\vec{w}}{p||\vec{w}||}$ và $\vec{w'_0} = \frac{2w_0}{p||\vec{w}||}$, ta có:
        phương trình đường thẳng $d, d_1, d_2$ được viết lại thành:

        $$\vec{w'}\vec{x} + w'_0 = 0 (d)$$
        $$\vec{w'}\vec{x} + w'_0 - 1 = 0 (d_1)$$
        $$\vec{w'}\vec{x} + w'_0 + 1 = 0 (d_2)$$

        Vậy ta có điều phải chứng minh.

        
        \item[3] Support Vector Machine
         
        \textbf{Bài toán}: Cho hai lớp và các điểm thuộc hai lớp đấy. 
        Gọi các điểm đấy là điểm dữ liệu.
        Cần tìm đường thẳng $d$ để khoảng cách giữa hai đường 
        thẳng song song, cách đều đường thẳng $d$ mà vùng
        không gian tạo bởi hai đường thẳng đó vẫn không 
        chứa bất kỳ điểm dữ liệu nào là lớn nhất.

        Theo mệnh đề $(**)$ ta có:\newline
        Chọn cách biểu diễn đường thẳng $d$ là $\vec{w}\vec{x} + w_0 = 0 (d)$
        sao cho hai đường thẳng song song và cách đều đường thẳng 
        $d$ có dạng:
        $$\vec{w}\vec{x} + w_0 - 1 = 0 (d_1)$$
        $$\vec{w}\vec{x} + w_0 + 1 = 0 (d_2)$$

        Khoảng cách giữa $d_1$ và $d_2$ là:
        $$\Delta _{12} = \frac{2}{||\vec{w}||}$$


        

        
        





    \end{description}
    % \item SVM
    
    % \end{enumerate}
\end{document}