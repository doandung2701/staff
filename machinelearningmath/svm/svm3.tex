\documentclass{article}
\usepackage[utf8]{vietnam}

\title{svm}
\author{cuongvomanh}
\date{March 2019}

\usepackage{natbib}
\usepackage{amsmath}

\begin{document}
    \section{Proven}
    \begin{enumerate}
        \item Phụ đề $(*)$
          
        Ta có:
        Vector từ điểm $O(0,0)$ đến mặt phẳng $\vec{w}\vec{x} + c = 0 (d)$ 
        hay vector từ điểm $O(0,0)$ đến hình chiếu của điểm $O$ lên 
        mặt phẳng $d$ là:
        $$\vec{(O,d)} = \frac{-c}{||\vec{w}||}$$

        Từ đây ta có:
        Vector từ điểm $I$ đến mặt phẳng $\vec{w}\vec{x} + c = 0 (d)$ là:
        
        $$\vec{(I,d)} = -\vec{(O,d')} + \vec{(O,d)}$$
        Với $\vec{w}(\vec{x} - \vec{I}) = 0 (d')$ 
        là mặt phẳng đi qua $I$ và song song với mặt phẳng $d$
        $$\Rightarrow \vec{(I,d)} = -\frac{\vec{w}\vec{I}}{||\vec{w}||} +  \frac{-c}{||\vec{w}||}$$
        $$\Rightarrow \vec{(I,d)} = -\frac{\vec{w}\vec{I}+c}{||\vec{w}||}$$

        Từ đó khoảng cách từ điểm $I$ đến mặt phẳng $\vec{w}\vec{x} + c = 0 (d)$ là:
        $$d(d,O) = ||\frac{\vec{w}\vec{I}+c}{||\vec{w}||}|| $$
        

        
        \item Phụ đề (**) \newline 
        \textbf{Xét bài toán: }
        Tìm các điểm cực trị của hàm số $w = f(x,y)$ (1.1)\newline 
        Với điều kiện ràng buộc $g(x,y) = b$ (1.2)\newline 
        Trong đó $x$ và $y$ là các biến chọn, $w$ là biến mục tiêu, $f$ là hàm mục tiêu \newline 
        Phương pháp nhân tử Lagrange: \newline 
        Xuất phát từ hàm mục tiêu $(1.1)$ và điều kiện $(1.2)$ ta lập hàm số sau (gọi là hàm
        Lagrange):
        $$L(x,y,\lambda) = f(x,y) + \lambda [b - g(x,y)] (1.3)$$
        Hàm số $(1.3)$ có thêm một biến phụ $\lambda$, gọi là nhân tử Lagrange. Chú ý rằng với 
        tất cả các điểm $M(x,y)$ thoản mãn điều kiện $(1.2)$, tức là khi xét cặp biến chọn (x,y) trong miền biến
        thiên đã bị thu hẹp bởi điều kiện $(1.2)$, hàm mục tiêu $w$ đồng nhất với hàm số L.
        Định lý sau đây cho biết mối liên hệ giữa hàm số Lagrange $(1.3)$ và bài toán cực trị có điều kiện mà ta
        đang xem xét. \newline 
        \textbf{Định lý: } \newline 
        Giả sử các hàm số $f(x,y)$ và $g(x,y)$ có đạo hàm riêng liên tục trong một lân cận của điểm $M_0(x_0,y_0)$
        và $g'_y(x_0, y_0) \neq 0$. Khi đó nếu hàm số $(1.1)$, với điều kiện $(1.2)$ đạt cực trị tại điểm 
        $M_0(x_0,y_0)$ thì tồn tại bộ ba số thực $x_0, y_0, \lambda_0)$ là nghiệm của hệ phương trình:
        $$
        \left\{\begin{matrix}
            L'_\lambda = b - g(x,y) \\
            L'x = f'_x - \lambda g'_x = 0 \\
            L'y = f'_y - \lambda g'_y = 0
        \end{matrix}\right.$$

        \textbf{Ý nghĩa của nhân tử Lagrange} \newline 
        Trong mô hình bài toán cực trị có điều kiện ta phân biệt vai trò của các biến số 
        $x_1, x_2, \dots x_n, w$ với tham số $b$. Các biến chọn  $x_1, x_2, \dots x_n$ và 
        biến mục tiêu $w$ được gọi là các biến nội sinh, do bản thân mô hình quyết định thông qua
        phương pháp nhân tử Lagrange. Khác với biến nội sinh, tham số b là một hằng số cho trước,
        không do mô hình quyết định. Phương án chọn tối ưu $\bar{X}(\bar{x_1}, \bar{x_2}, \dots \bar{x_n})$
        của bài toán và giá trị tối ưu $\bar{w}$ của hàm mục tiêu phụ thuộc vào $b$. \newline 
        $\bar{x_1} = \bar{x_1}(b), \bar{x_2} = \bar{x_2}(b), \dots , \bar{x_n} = \bar{x_n}(b)$ \newline 
        Theo phương pháp nhân tử Lagrange, phương án chọn tối ưu nói trên được xác định cùng với một 
        giá trị của nhân tử Lagrange $\bar{\lambda} = \bar{\lambda}(b)$ và giá trị tối ưu của $w$ là 
        hàm số của $b$ \newline 
        $\bar{w} = f(x_1, x_2, \dots , x_n) = \bar{w}(b)$ \newline 
        Theo quy tắc đạo hàm hợp ta có: \newline 
        $$\frac{d\bar{w}}{db} = \frac{\delta \bar{f}}{\delta x_1} \frac{d \bar{x_1}}{db}
        + \frac{\delta \bar{f}}{\delta x_2} \frac{d \bar{x_2}}{db} 
        + \dots + \frac{\delta \bar{f}}{\delta x_n} \frac{d \bar{x_n}}{db}
        $$
        Do $(\bar{\lambda},\bar{x_1}, \bar{x_2}, \dots \bar{x_n})$ là điểm dừng của hàm số Lagrange nên ta có: \newline
        $\frac{\delta \bar{f}}{\delta x_i} = \lambda \frac{\delta \bar{g}}{\delta x_i} \forall i \in 1,2, \dots n$ \newline 
        Vậy $$\frac{d\bar{w}}{db} = \lambda \left\{   \frac{\delta \bar{g}}{\delta x_1} \frac{d \bar{x_1}}{db}
        + \frac{\delta \bar{g}}{\delta x_2} \frac{d \bar{x_2}}{db} 
        + \dots + \frac{\delta \bar{g}}{\delta x_n} \frac{d \bar{x_n}}{db} \right \}
         (1.4)$$ 
        Mà ta lại có $ g(x_1, x_2, \dots , x_n) = 0$. Lấy đạo hàm hai vế của đồng nhất thức này theo $b$ ta có:
        $$
        \frac{\delta \bar{g}}{\delta x_1} \frac{d \bar{x_1}}{db}
        + \frac{\delta \bar{g}}{\delta x_2} \frac{d \bar{x_2}}{db} 
        + \dots + \frac{\delta \bar{g}}{\delta x_n} \frac{d \bar{x_n}}{db} = 1 (1.5)
        $$
        Từ $(1.4)$ và $(1.5)$ suy ra: $\bar{\lambda} = \frac{d\bar{w}}{db} (1.6)$ \newline 
        Hệ thức trên cho thấy nhân tử Lagrange $\bar{\lambda}$ chính là giá trị $\bar{w}$ cận biên của $b$.
        Điều này có nghãi là khi b tăng thêm 1 đơn vị thì giá trị tối ưu $\bar{w}$ của hàm mục tiêu thay đổi 
        một lượng xấp xỉ bằng $\bar{\lambda}$.


        \textbf{Bài toán}
        Chọn $(x,y)$ để hàm số $w = f(x,y)$ đạt giá trị cực đại (cực tiểu) với 
        điều kiện $g(x,y) \geq b$ [hoặc $ g(x,y) \leq b$] \newline 
        \textbf{Phương pháp chung:} Để giải bài toán này, trước hết ta thay điều 
        kiện bằng phương trình $g(x,y) = b$. Bằng phương pháp nhân tử Lagrange ta
        tìm được $x=\bar{x}, y = \bar{y}$ và $\lambda = \bar{\lambda}$.
        Quyết định cuối cùng dựa theo nguyên tắc sau đây: \newline 
        \begin{enumerate}
            \item Đối với bài toán cực đại hóa tham số $w = f(x, y)$ với điều kiện $g(x,y) \geq b$
            \begin{itemize}
                \item Nếu $\bar{\lambda} < 0$ thì điều kiện áp đặt thực sự là ràng 
                buộc và phương án chọn tối ưu là $x=\bar{x}, y = \bar{y}$
                \item $\bar{\lambda} \geq 0$ thì điều kiện áp đặt không phải là  ràng 
                buộc thực sự. Trong trường hợp này ta bỏ điều kiện và quay sang giải bài toán 
                cực trị không có điều kiện.
            \end{itemize}

            \item Đối với bài toán cực đại hóa tham số $w = f(x, y)$ với điều kiện $g(x,y) \leq b$ 
            \begin{itemize}
                \item Nếu $\bar{\lambda} \geq 0$ thì điều kiện áp đặt thực sự là ràng 
                buộc và phương án chọn tối ưu là $x=\bar{x}, y = \bar{y}$
                \item $\bar{\lambda} < 0$ thì điều kiện áp đặt không phải là ràng 
                buộc thực sự. Trong trường hợp này ta bỏ điều kiện và quay sang giải bài toán 
                cực trị không có điều kiện.
            \end{itemize}

            \item Đối với bài toán cực tiểu hóa tham số $w = f(x, y)$ với điều kiện $g(x,y) \geq b$ 
            \begin{itemize}
                \item Nếu $\bar{\lambda} \geq 0$ thì điều kiện áp đặt thực sự là ràng 
                buộc và phương án chọn tối ưu là $x=\bar{x}, y = \bar{y}$
                \item $\bar{\lambda} < 0$ thì điều kiện áp đặt không phải là ràng 
                buộc thực sự. Trong trường hợp này ta bỏ điều kiện và quay sang giải bài toán 
                cực trị không có điều kiện.
            \end{itemize}

            \item Đối với bài toán cực tiểu hóa tham số $w = f(x, y)$ với điều kiện $g(x,y) \leq b$ 
            \begin{itemize}
                \item Nếu $\bar{\lambda} < 0$ thì điều kiện áp đặt thực sự là ràng 
                buộc và phương án chọn tối ưu là $x=\bar{x}, y = \bar{y}$
                \item $\bar{\lambda} \geq 0$ thì điều kiện áp đặt không phải là ràng 
                buộc thực sự. Trong trường hợp này ta bỏ điều kiện và quay sang giải bài toán 
                cực trị không có điều kiện.
            \end{itemize}

        \end{enumerate}
        
        
        \item Support Vector Machine
        
        
         
        \textbf{Bài toán}: Cho hai lớp và các điểm thuộc hai lớp đấy. 
        Gọi các điểm đấy là các điểm dữ liệu.
        Cần tìm mặt phẳng $d$ để khoảng cách giữa hai mặt 
        phẳng song song, cách đều mặt phẳng $d$ mà vùng
        không gian tạo bởi hai mặt phẳng đó vẫn không 
        chứa bất kỳ điểm dữ liệu nào là lớn nhất.

        Bài toán trên tương đương với bài toán: \newline
        \textbf{Bài toán} Cho hai lớp và các điểm thuộc hai lớp đấy. Gọi các điểm đấy là các điểm dữ liệu.
        Cần tìm hai mặt phẳng song song sao cho vùng không gian tạo bởi hai mặt 
        phẳng đấy chia không gian còn lại thành hai phần khác nhau, mỗi phần chỉ chứa 
        các điểm dữ liệu của một lớp mà khoảng cách giữa hai mặt phẳng đấy là lớn nhất.

        Ta có:
        Bất kỳ hai mặt phẳng song song nào đều được biểu diễn bởi:
        $$\vec{w}\vec{x} + c - 1 = 0 (d_1)$$
        $$\vec{w}\vec{x} + c + 1 = 0 (d_2)$$

        Khoảng cách giữa $d_1$ và $d_2$ là:
        $$\Delta _{12} = \frac{2}{||\vec{w}||}$$
        Giả sử tập các điểm dữ liệu là:
        $$D =  {(\vec{x_1},y_1), (\vec{x_2}, y_2), \dots, (\vec{x_n}, y_n)}$$
        với $n$ là số lượng dữ liệu.\newline
        Bài toán đã cho được viết lại thành:\newline
        \textbf{Bài toán} Cho tập $D =\left \{ (\vec{x_1},y_1), (\vec{x_2}, y_2), \dots, (\vec{x_n}, y_n) \right \}  $ \newline 
        Tìm giá trị của $\vec{w}, c$ để $\frac{||\vec{w}||^{2}}{2}$ nhỏ nhất với điều kiện:
        $$
        \left\{\begin{matrix}
            \vec{w}\vec{x_i} + c \geq 1, \text{nếu } y_i = 1 \\
            \vec{w}\vec{x_i} + c \leq 1, \text{nếu } y_i = -1
        \end{matrix}\right.$$
        với $\forall (x_i, y_i) \in D$

        Viết gọn điều kiện lại, ta có bài toán:\newline 
        \textbf{Bài toán} Cho tập $D =\left \{  (\vec{x_1},y_1), (\vec{x_2}, y_2), \dots, (\vec{x_n}, y_n) \right \}$ \newline 
        Tìm giá trị của $\vec{w}, c$ để $\frac{||\vec{w}||^{2}}{2}$ nhỏ nhất với điều kiện: 
        $y_i(\vec{w}\vec{x_i} + c) \geq 1$ \newline 
        với $\forall (x_i, y_i) \in D$ \newline 

        \textbf{Giải: } \newline 
        Xét hàm số Lagrange:
        $$L(\vec{w}, c, \lambda) = \frac{||\vec{w}||^{2}}{2} + \sum_{i=1}^{n}{\lambda _i[1 - y_i(\vec{w}\vec{x_i} + c)]}$$
        Viết $\lambda$ thay thế cho $n$ biến $\lambda _i$ với $i \in 1 \dots n$ \newline 


        Ta tiếp cận bài toán sau trước: \newline 
        Cho tập $D =\left \{  (\vec{x_1},y_1), (\vec{x_2}, y_2), \dots, (\vec{x_n}, y_n) \right \}$ \newline 
        Tìm giá trị của $\vec{w}, c$ để $\frac{||\vec{w}||^{2}}{2}$ nhỏ nhất với điều kiện: 
        $y_i(\vec{w}\vec{x_i} + c) = 1$ \newline 
        với $\forall (x_i, y_i) \in D$ \newline
        Ta tiếp cận bằng phương pháp Lagrange: \newline 
        Giả sử các hàm số 
        $f(\vec{w},c) = \frac{||\vec{w}||}{2} $ và $g_i(\vec{w},c) = y_i(\vec{w}\vec{x_i} + c) \forall i \in 1 \dots n$ 
        có đạo hàm riêng
        cấp hai liên tục. Khi đó nếu hàm số $f(\vec{w},c)$, với điều kiện $g_i(\vec{w},c) = 1$ đạt cực trị tại điểm 
        $M_0(\vec{\bar{w}},\bar{c})$ thì tồn tại bộ $(\vec{\bar{w}}, \bar{c}, \bar{\lambda})$ là nghiệm của hệ phương trình:
        $$
        \left\{\begin{matrix}
            L'_{\lambda _i} = 1 - y_i(\vec{w}\vec{x_i} + c) = 0 \forall i \in 1 \dots n\\
            L'_{\vec{w}} = \vec{w} - \sum_{i=1}^{n}{\lambda _i y_i \vec{x_i}} = 0 \\
            L'_c = -\sum_{i=1}^{n}{\lambda _i y_i} = 0
        \end{matrix}\right.$$
        
        Quay lại bài toán ban đầu: \newline 
        Giả sử $(\vec{\bar{w}}, \bar{c}, \bar{\lambda})$ là một nghiệm của hệ phương trình trên. \newline 
        Xét các biến $\bar{\lambda}(\bar{\lambda} _1, \bar{\lambda} _2, \dots \bar{\lambda} _n)$ \newline 
        Giả sử $P, Q$ là tập con của $\bar{\lambda}$ sao cho $\bar{\lambda}_i \geq 0 \forall \bar{\lambda}_i \in P$
        và $\bar{\lambda}_i < 0 \forall \bar{\lambda}_i \in Q$
        
        % Theo phụ đề $(**)$ ta chỉ xét những $\lambda \geq 0$. Để 
        Theo ý nghĩa của nhân tử Lagrange trong phụ đề $(**)$ ta có $\forall \bar{\lambda}_i \in \bar{\lambda}$ ta có:
        $\bar{\lambda_i} = \frac{d\bar{f}}{db_i}$.  \newline 
        Vậy ta có $\forall \bar{\lambda}_i \in Q$ thì khi $b_i$ tăng thì $f$
        sẽ giảm tại lân cận của $b_i$ hay $f$ không đạt giá trị nhỏ nhất tại biên $g_i(\vec{w},c) = b_i$. Vậy ta sẽ bỏ qua 
        điều kiện $g_i(\vec{w},c) \geq b_i$ để tìm giá trị nhỏ nhất của $f$ (điều kiện cần để tại điểm $M_0(\vec{\bar{w}},\bar{c})$
        $f$ lớn nhât với điều kiện $g_i(\vec{w},c) \geq b_i$ là tại $M_0(\vec{\bar{w}},\bar{c})$
        $f$ lớn nhât không có điều kiện $g_i(\vec{w},c) \geq b_i$) \newline 
        
        Và ta cũng có $\forall \bar{\lambda}_i \in P$ thì khi $b_i$ giảm thì $f$
        sẽ giảm tại lân cận của $b_i$ hay $f$ đạt giá trị nhỏ nhất với điều kiện $g_i(\vec{w},c) \geq b_i$ tại điểm 
        thỏa mãn điều kiện $g_i(\vec{w},c) = b_i$

        Theo đó để giải quyết bài tóan ban đầu, ta chuyển sang giải quyết bài toán sau: \newline 
        Cho tập $D =\left \{  (\vec{x_1},y_1), (\vec{x_2}, y_2), \dots, (\vec{x_n}, y_n) \right \}$ \newline
        và giả sử $\bar{\lambda}(\bar{\lambda} _1, \bar{\lambda} _2, \dots \bar{\lambda} _n)$ là điểm để $f$
        đạt cực tiểu với điều kiện $y_i(\vec{w}\vec{x_i} + c) \geq 1$ với $\forall (x_i, y_i) \in D$. \newline 
        Giả sử $P$ là tập con của $\bar{\lambda}$ sao cho $\bar{\lambda}_i \geq 0 \forall \bar{\lambda}_i \in P$ \newline 
        Giả sử tập $D_1 \in D$ sao cho $\bar{\lambda}_i$ tương ứng với $(\vec{x_i},y_i)$ thỏa mãn thuộc tập $P$.
        Tìm giá trị của $\vec{w}, c$ để $\frac{||\vec{w}||^{2}}{2}$ nhỏ nhất với điều kiện: 
        $y_i(\vec{w}\vec{x_i} + c) = 1$ \newline 
        với $\forall (x_i, y_i) \in D_1$ \newline 

        Xem tập $D_1$ như tập $D$ ở trên, tương tự như trên ta có: \newline 
        Xét hàm số Lagrange:
        $$L(\vec{w}, c, \lambda) = \frac{||\vec{w}||^{2}}{2} + \sum_{\forall (x_i, y_i) \in D_1}^{}{\lambda _i[1 - y_i(\vec{w}\vec{x_i} + c)]}$$

        Ta xét bài toán: \newline 
        Tìm giá trị lớn nhất của:
        $$R(\lambda) = \frac{||\vec{w}||^{2}}{2}$$
        Với điều kiện:
        $$\left\{\begin{matrix}
            y_i(\vec{w}\vec{x_i} + c) = 1 \forall (x_i, y_i) \in D_1\\
            \vec{w} = \sum_{\forall (x_i, y_i) \in D_1}^{}{\lambda _i y_i \vec{x_i}}  \\
            \sum_{\forall (x_i, y_i) \in D_1}^{}{\lambda _i y_i} = 0 \\
            $\lambda_i \geq 0 \forall \lambda_i \in P$
        \end{matrix}\right.$$
        
        Giả sử tại $\bar{\lambda}$ $L(\vec{w}, c, \lambda)$ đạt cực tiểu, ta sẽ chứng minh rằng tại $\bar{\lambda}$
        thì $R(\lambda)$ đạt cực đại. Thật vậy giả sử tại $\bar{\lambda}$ $R(\lambda)$ không đạt cực đại.
        Vậy tồn tại một $\bar{\lambda^{*}}$ sao cho:
        % tại $\bar{\lambda^{*}}$ $L(\vec{w}, c, \lambda^{*}) = L_{min}$ đạt cực tiểu và
        $$R(\bar{\lambda^{*}}) > R(\bar{\lambda})$$ và
        $$ R(\bar{\lambda^{*}}) = L(\vec{w}, c, \lambda^{*})$$
        % Mà tại $\bar{\lambda}$ $L(\vec{w}, c, \lambda)$ đạt cực tiểu:
        % $$\Rightarrow R(\bar{\lambda}) = L(\vec{w}, c, \lambda)$$
        Mà: 
        $$ R(\bar{\lambda^{*}}) < L(\vec{w}, c, \lambda)$$
        $$ \Rightarrow L(\vec{w}, c, \lambda^{*}) < L(\vec{w}, c, \lambda)$$
        Mâu thuẫn với giả thiết tại $\bar{\lambda}$ $L(\vec{w}, c, \lambda)$ đạt cực tiểu

        % Giả sử các hàm số 
        % $f(\vec{w},c) = \frac{||\vec{w}||}{2} $ và $g_i(\vec{w},c) = y_i(\vec{w}\vec{x_i} + c) \forall (x_i, y_i) \in D_1$ 
        % có đạo hàm riêng
        % cấp hai liên tục. Khi đó nếu hàm số $f(\vec{w},c)$, với điều kiện $g_i(\vec{w},c) = 1$ đạt cực trị tại điểm 
        % $M_0(\vec{\bar{w}},\bar{c})$ thì tồn tại bộ $(\vec{\bar{w}}, \bar{c}, \bar{\lambda})$ là nghiệm của hệ phương trình:
        % $$
        % \left\{\begin{matrix}
        %     L'_{\lambda _i} = 1 - y_i(\vec{w}\vec{x_i} + c) = 0 \forall (x_i, y_i) \in D_1\\
        %     L'_{\vec{w}} = \vec{w} - \sum_{\forall (x_i, y_i) \in D_1}^{}{\lambda _i y_i \vec{x_i}} = 0 \\
        %     L'_c = -\sum_{\forall (x_i, y_i) \in D_1}^{}{\lambda _i y_i} = 0
        % \end{matrix}\right.$$
        % Hay tương đương với:
        % $$\left\{\begin{matrix}
        %     y_i(\vec{w}\vec{x_i} + c) = 1 \forall (x_i, y_i) \in D_1\\
        %     \vec{w} = \sum_{\forall (x_i, y_i) \in D_1}^{}{\lambda _i y_i \vec{x_i}}  \\
        %     \sum_{\forall (x_i, y_i) \in D_1}^{}{\lambda _i y_i} = 0
        % \end{matrix}\right.$$
        % Giả sử $(\vec{\bar{w}}, \bar{c}, \bar{\lambda})$ là một nghiệm của hệ phương trình trên.\newline 
        % Khi đó 
        % % $$L(\vec{w}, c, \lambda) =  \frac{||\vec{w}||^{2}}{2} $$
        % % $$\Leftrightarrow  L(\vec{w}, c, \lambda) =  \frac{(\sum_{\forall (x_i, y_i) \in D_1}^{}{\lambda _i y_i \vec{x_i}})^{2}}{2} (2.1)$$
        % % Ta sẽ tìm giá trị nhỏ nhất của $(2.1)$ với điều kiện $\sum_{\forall (x_i, y_i) \in D_1}^{}{\lambda _i y_i} = 0$ và $\lambda_i \geq 0 \forall \lambda_i \in P$

        % $$L(\vec{w}, c, \lambda) =  \frac{||\vec{w}||^{2}}{2} + \sum_{\forall (x_i, y_i) \in D_1}^{}{\lambda _i} - \sum_{\forall (x_i, y_i) \in D_1}^{}{\lambda _i y_i c} - \vec{w}\sum_{\forall (x_i, y_i) \in D_1}^{}{\lambda _i y_i \vec{x_i}} $$
        % $$L(\vec{w}, c, \lambda) =  -\frac{(\sum_{\forall (x_i, y_i) \in D_1}^{}{\lambda _i y_i \vec{x_i}})^{2}}{2} + \sum_{\forall (x_i, y_i) \in D_1}^{}{\lambda _i} $$


        









    \end{enumerate}
    % \item SVM
    
    % \end{enumerate}
\end{document}